\hypertarget{compile_windows_asio_msvc_comp_win_asiomsvc1}{}\section{Portaudio Windows A\+S\+I\+O with M\+S\+VC}\label{compile_windows_asio_msvc_comp_win_asiomsvc1}
This tutorial describes how to build Port\+Audio with A\+S\+IO support using M\+S\+VC {\itshape from scratch}, without an existing Visual Studio project. For instructions for building Port\+Audio (including A\+S\+IO support) using the bundled Visual Studio project file see the compiling instructions for \hyperlink{compile_windows}{Building Port\+Audio for Windows using Microsoft Visual Studio}.

A\+S\+IO is a low latency audio A\+PI from Steinberg. To compile an A\+S\+IO application, you must first download the A\+S\+IO S\+DK from Steinberg. You also need to obtain A\+S\+IO drivers for your audio device. Download the A\+S\+IO S\+DK from Steinberg at \href{http://www.steinberg.net/en/company/developer.html}{\tt http\+://www.\+steinberg.\+net/en/company/developer.\+html} . The S\+DK is free but you will need to set up a developer account with Steinberg.

This tutorial assumes that you have 3 directories set up at the same level (side by side), one containing Port\+Audio, one containing the A\+S\+IO S\+DK and one containing your Visual Studio project\+:


\begin{DoxyCode}
/ASIOSDK2 
/\hyperlink{namespaceportaudio}{portaudio}
/DirContainingYourVisualStudioProject  (should directly contain \hyperlink{nasm_8h_aaf57ffcfef0bee04f46c6ada2a905a8c}{the} .sln, .vcproj or .vcprojx etc.)
\end{DoxyCode}


First, make sure that the Steinberg S\+DK and the portaudio files are \char`\"{}side by side\char`\"{} in the same directory.

Open Microsoft Visual C++ and create a new blank Console exe Project/\+Workspace in that same directory.

For example, the paths for all three groups might read like this\+:


\begin{DoxyCode}
C:\(\backslash\)Program Files\(\backslash\)Microsoft Visual Studio\(\backslash\)VC98\(\backslash\)My Projects\(\backslash\)ASIOSDK2
C:\(\backslash\)Program Files\(\backslash\)Microsoft Visual Studio\(\backslash\)VC98\(\backslash\)My Projects\(\backslash\)portaudio
C:\(\backslash\)Program Files\(\backslash\)Microsoft Visual Studio\(\backslash\)VC98\(\backslash\)My Projects\(\backslash\)Sawtooth
\end{DoxyCode}


Next, add the following Steinberg A\+S\+IO S\+DK files to the project Source Files\+:


\begin{DoxyCode}
asio.cpp                        (ASIOSDK2\(\backslash\)common)
asiodrivers.cpp                 (ASIOSDK2\(\backslash\)host)
asiolist.cpp                    (ASIOSDK2\(\backslash\)host\(\backslash\)pc)
\end{DoxyCode}


Then, add the following Port\+Audio files to the project Source Files\+:


\begin{DoxyCode}
pa\_asio.cpp                     (\hyperlink{namespaceportaudio}{portaudio}\(\backslash\)src\(\backslash\)hostapi\(\backslash\)asio)
pa\_allocation.c                 (\hyperlink{namespaceportaudio}{portaudio}\(\backslash\)src\(\backslash\)common)
pa\_converters.c                 (\hyperlink{namespaceportaudio}{portaudio}\(\backslash\)src\(\backslash\)common)
pa\_cpuload.c                    (\hyperlink{namespaceportaudio}{portaudio}\(\backslash\)src\(\backslash\)common)
pa\_dither.c                     (\hyperlink{namespaceportaudio}{portaudio}\(\backslash\)src\(\backslash\)common)
pa\_front.c                      (\hyperlink{namespaceportaudio}{portaudio}\(\backslash\)src\(\backslash\)common)
pa\_process.c                    (\hyperlink{namespaceportaudio}{portaudio}\(\backslash\)src\(\backslash\)common)
pa\_ringbuffer.c                 (\hyperlink{namespaceportaudio}{portaudio}\(\backslash\)src\(\backslash\)common)
pa\_stream.c                     (\hyperlink{namespaceportaudio}{portaudio}\(\backslash\)src\(\backslash\)common)
pa\_trace.c                      (\hyperlink{namespaceportaudio}{portaudio}\(\backslash\)src\(\backslash\)common)
pa\_win\_hostapis.c               (\hyperlink{namespaceportaudio}{portaudio}\(\backslash\)src\(\backslash\)os\(\backslash\)win)
pa\_win\_util.c                   (\hyperlink{namespaceportaudio}{portaudio}\(\backslash\)src\(\backslash\)os\(\backslash\)win)
pa\_win\_coinitialize.c           (\hyperlink{namespaceportaudio}{portaudio}\(\backslash\)src\(\backslash\)os\(\backslash\)win)
pa\_win\_waveformat.c             (\hyperlink{namespaceportaudio}{portaudio}\(\backslash\)src\(\backslash\)os\(\backslash\)win)
pa\_x86\_plain\_converters.c       (\hyperlink{namespaceportaudio}{portaudio}\(\backslash\)src\(\backslash\)os\(\backslash\)win)
paex\_saw.c                      (\hyperlink{namespaceportaudio}{portaudio}\(\backslash\)examples)  (Or another file containing 
      \hyperlink{elements_8c_a0ddf1224851353fc92bfbff6f499fa97}{main}() 
                                                      \hyperlink{hashrout_8h_af69473f95324d0c0f91fdfb1d1a00360}{for} \hyperlink{nasm_8h_aaf57ffcfef0bee04f46c6ada2a905a8c}{the} console exe to be built.)
\end{DoxyCode}


Although not strictly necessary, you may also want to add the following files to the project Header Files\+:


\begin{DoxyCode}
\hyperlink{namespaceportaudio}{portaudio}.h                     (\hyperlink{namespaceportaudio}{portaudio}\(\backslash\)include)
pa\_asio.h                       (\hyperlink{namespaceportaudio}{portaudio}\(\backslash\)include)
\end{DoxyCode}


These header files define the interfaces to the Port\+Audio A\+PI.

Next, go to Project Settings $>$ All Configurations $>$ C/\+C++ $>$ Preprocessor $>$ Preprocessor Definitions and add P\+A\+\_\+\+U\+S\+E\+\_\+\+A\+S\+IO=1 to any entries that might be there.

eg\+: W\+I\+N32;\+\_\+\+C\+O\+N\+S\+O\+LE;\+\_\+\+M\+B\+CS changes to W\+I\+N32;\+\_\+\+C\+O\+N\+S\+O\+LE,\+\_\+\+M\+B\+CS;P\+A\+\_\+\+U\+S\+E\+\_\+\+A\+S\+IO=1

Then, on the same Project Settings tab, go down to Additional Include Directories (in V\+S2010 you\textquotesingle{}ll find this setting under C/\+C++ $>$ General) and enter the following relative include paths\+:


\begin{DoxyCode}
..\(\backslash\)portaudio\(\backslash\)include;..\(\backslash\)portaudio\(\backslash\)src\(\backslash\)common;..\(\backslash\)portaudio\(\backslash\)src\(\backslash\)os\(\backslash\)win;..\(\backslash\)asiosdk2\(\backslash\)common;..\(\backslash\)asiosdk2\(\backslash\)host;..
      \(\backslash\)asiosdk2\(\backslash\)host\(\backslash\)pc
\end{DoxyCode}


You\textquotesingle{}ll need to make sure the relative paths are correct for the particular directory layout you\textquotesingle{}re using. The above should work fine if you use the side-\/by-\/side layout we recommended earlier.

Some source code in the A\+S\+IO S\+DK is not compatible with the Win32 A\+PI U\+N\+I\+C\+O\+DE mode (The A\+S\+IO S\+DK expects the non-\/\+Unicode Win32 A\+PI). Therefore you need to make sure your project is set to not use Unicode. You do this by setting the project Character Set to \char`\"{}\+Use Multi-\/\+Byte Character Set\char`\"{} (N\+OT \char`\"{}\+Use Unicode Character Set\char`\"{}). In V\+S2010 the Character Set option can be found at Configuration Properties $>$ General $>$ Character Set. (An alternative to setting the project to non-\/\+Unicode is to patch asiolist.\+cpp to work when U\+N\+I\+C\+O\+DE is defined\+: put \#undef U\+N\+I\+C\+O\+DE at the top of the file before windows.\+h is included.)

You should now be able to build any of the test executables in the portaudio\textbackslash{}examples directory. We suggest that you start with paex\+\_\+saw.\+c because it\textquotesingle{}s one of the simplest example files.

--- Chris Share, Tom Mc\+Candless, Ross Bencina

Back to the Tutorial\+: \hyperlink{tutorial_start}{Port\+Audio Tutorials} 