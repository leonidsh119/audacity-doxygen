\hypertarget{index_ffmpeg_intro}{}\section{Introduction}\label{index_ffmpeg_intro}
This document describes the usage of the different libraries provided by F\+Fmpeg.

\begin{DoxyItemize}
\item \hyperlink{group__libavc}{libavcodec} encoding/decoding library \item libavfilter graph-\/based frame editing library \item \hyperlink{group__libavf}{libavformat} I/O and muxing/demuxing library \item libavdevice special devices muxing/demuxing library \item \hyperlink{group__lavu}{libavutil} common utility library \item libswresample audio resampling, format conversion and mixing \item libpostproc post processing library \item libswscale color conversion and scaling library\end{DoxyItemize}
\hypertarget{index_ffmpeg_versioning}{}\section{Versioning and compatibility}\label{index_ffmpeg_versioning}
Each of the F\+Fmpeg libraries contains a version.\+h header, which defines a major, minor and micro version number with the {\itshape L\+I\+B\+R\+A\+R\+Y\+N\+A\+M\+E\+\_\+\+V\+E\+R\+S\+I\+O\+N\+\_\+\{M\+A\+J\+OR,M\+I\+N\+OR,M\+I\+C\+RO\}} macros. The major version number is incremented with backward incompatible changes -\/ e.\+g. removing parts of the public A\+PI, reordering public struct members, etc. The minor version number is incremented for backward compatible A\+PI changes or major new features -\/ e.\+g. adding a new public function or a new decoder. The micro version number is incremented for smaller changes that a calling program might still want to check for -\/ e.\+g. changing behavior in a previously unspecified situation.

F\+Fmpeg guarantees backward A\+PI and A\+BI compatibility for each library as long as its major version number is unchanged. This means that no public symbols will be removed or renamed. Types and names of the public struct members and values of public macros and enums will remain the same (unless they were explicitly declared as not part of the public A\+PI). Documented behavior will not change.

In other words, any correct program that works with a given F\+Fmpeg snapshot should work just as well without any changes with any later snapshot with the same major versions. This applies to both rebuilding the program against new F\+Fmpeg versions or to replacing the dynamic F\+Fmpeg libraries that a program links against.

However, new public symbols may be added and new members may be appended to public structs whose size is not part of public A\+BI (most public structs in F\+Fmpeg). New macros and enum values may be added. Behavior in undocumented situations may change slightly (and be documented). All those are accompanied by an entry in doc/\+A\+P\+Ichanges and incrementing either the minor or micro version number. 