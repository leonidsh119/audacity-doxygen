The next step is to open a stream, which is similar to opening a file. You can specify whether you want audio input and/or output, how many channels, the data format, sample rate, etc. Opening a \textquotesingle{}\textquotesingle{}default\textquotesingle{}\textquotesingle{} stream means opening the default input and output devices, which saves you the trouble of getting a list of devices and choosing one from the list. (We\textquotesingle{}ll see how to do that later.) 
\begin{DoxyCode}
\textcolor{preprocessor}{#define SAMPLE\_RATE (44100)}
\textcolor{keyword}{static} \hyperlink{structpa_test_data}{paTestData} \hyperlink{lib_2expat_8h_ac39e72a1de1cb50dbdc54b08d0432a24}{data};

.....

    \hyperlink{portaudio_8h_a19874734f89958fccf86785490d53b4c}{PaStream} *stream;
    \hyperlink{portaudio_8h_a4949e4a8ef9f9dbe8cbee414ce69841d}{PaError} err;

    \textcolor{comment}{/* Open an audio I/O stream. */}
    err = \hyperlink{portaudio_8h_a0a12735ac191200f696a43b87667b714}{Pa\_OpenDefaultStream}( &stream,
                                0,          \textcolor{comment}{/* no input channels */}
                                2,          \textcolor{comment}{/* stereo output */}
                                \hyperlink{portaudio_8h_a2f16d29916725b8791eae60ab9e0b081}{paFloat32},  \textcolor{comment}{/* 32 bit floating point output */}
                                \hyperlink{sine_8cxx_a254783139625fe796c8924e0b04b1aee}{SAMPLE\_RATE},
                                256,        \textcolor{comment}{/* frames per buffer, i.e. the number}
\textcolor{comment}{                                                   of sample frames that PortAudio will}
\textcolor{comment}{                                                   request from the callback. Many apps}
\textcolor{comment}{                                                   may want to use}
\textcolor{comment}{                                                   paFramesPerBufferUnspecified, which}
\textcolor{comment}{                                                   tells PortAudio to pick the best,}
\textcolor{comment}{                                                   possibly changing, buffer size.*/}
                                patestCallback, \textcolor{comment}{/* this is your callback function */}
                                &data ); \textcolor{comment}{/*This is a pointer that will be passed to}
\textcolor{comment}{                                                   your callback*/}
    \textcolor{keywordflow}{if}( err != \hyperlink{portaudio_8h_a2e45bf8b5145f131a91c128af2bdaec7aeb09d15a48b6c1034728a9c518cfe4ba}{paNoError} ) \textcolor{keywordflow}{goto} \hyperlink{sndfile__load_8m_a3a81ee160b23b01b04b7c08b35123218}{error};
\end{DoxyCode}


The data structure and callback are described in \hyperlink{writing_a_callback}{Writing a Callback Function}.

The above example opens the stream for writing, which is sufficient for playback. It is also possible to open a stream for reading, to do recording, or both reading and writing, for simultaneous recording and playback or even real-\/time audio processing. If you plan to do playback and recording at the same time, open only one stream with valid input and output parameters.

There are some caveats to note about simultaneous read/write\+:


\begin{DoxyItemize}
\item Some platforms can only open a read/write stream using the same device.
\item Although multiple streams can be opened, it is difficult to synchronize them.
\item Some platforms don\textquotesingle{}t support opening multiple streams on the same device.
\item Using multiple streams may not be as well tested as other features.
\item The Port\+Audio library calls must be made from the same thread or synchronized by the user.
\end{DoxyItemize}

Previous\+: \hyperlink{initializing_portaudio}{Initializing Port\+Audio} $\vert$ Next\+: \hyperlink{start_stop_abort}{Starting, Stopping and Aborting a Stream} 