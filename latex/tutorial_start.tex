These tutorials takes you through a hands-\/on example of using Port\+Audio to make sound. If you\textquotesingle{}d prefer to start with a top-\/down overview of the Port\+Audio A\+PI, check out the \hyperlink{api_overview}{Port\+Audio A\+PI Overview}.\hypertarget{tutorial_start_tut_start1}{}\section{Downloading}\label{tutorial_start_tut_start1}
First thing you need to do is download the Port\+Audio source code either \href{http://www.portaudio.com/download.html}{\tt as a tarball from the website}, or \href{http://www.portaudio.com/usingsvn.html}{\tt from the Subversion Repository}.\hypertarget{tutorial_start_tut_start2}{}\section{Compiling}\label{tutorial_start_tut_start2}
Once you\textquotesingle{}ve downloaded Port\+Audio you\textquotesingle{}ll need to compile it, which of course, depends on your environment\+:


\begin{DoxyItemize}
\item Windows
\begin{DoxyItemize}
\item \hyperlink{compile_windows}{Building Port\+Audio for Windows using Microsoft Visual Studio}
\item \hyperlink{compile_windows_mingw}{Building Portaudio for Windows with Min\+GW}
\item \hyperlink{compile_windows_asio_msvc}{Building Portaudio for Windows with A\+S\+IO support using M\+S\+VC}
\item \hyperlink{compile_cmake}{Creating M\+S\+VC Build Files via C\+Make}
\end{DoxyItemize}
\item Mac OS X
\begin{DoxyItemize}
\item \hyperlink{compile_mac_coreaudio}{Building Portaudio for Mac OS X}
\end{DoxyItemize}
\item P\+O\+S\+IX
\begin{DoxyItemize}
\item \hyperlink{compile_linux}{Building Portaudio for Linux}
\end{DoxyItemize}
\end{DoxyItemize}

Many platforms with G\+C\+C/make can use the simple ./configure \&\& make combination and simply use the resulting libraries in their code.\hypertarget{tutorial_start_tut_start3}{}\section{Programming with Port\+Audio}\label{tutorial_start_tut_start3}
Below are the steps to writing a Port\+Audio application using the callback technique\+:


\begin{DoxyItemize}
\item Write a callback function that will be called by Port\+Audio when audio processing is needed.
\item Initialize the PA library and open a stream for audio I/O.
\item Start the stream. Your callback function will be now be called repeatedly by PA in the background.
\item In your callback you can read audio data from the input\+Buffer and/or write data to the output\+Buffer.
\item Stop the stream by returning 1 from your callback, or by calling a stop function.
\item Close the stream and terminate the library.
\end{DoxyItemize}

In addition to this \char`\"{}\+Callback\char`\"{} architecture, V19 also supports a \char`\"{}\+Blocking I/\+O\char`\"{} model which uses read and write calls which may be more familiar to non-\/audio programmers. Note that at this time, not all A\+P\+Is support this functionality.

In this tutorial, we\textquotesingle{}ll show how to use the callback architecture to play a sawtooth wave. Much of the tutorial is taken from the file paex\+\_\+saw.\+c, which is part of the Port\+Audio distribution. When you\textquotesingle{}re done with this tutorial, you\textquotesingle{}ll be armed with the basic knowledge you need to write an audio program. If you need more sample code, look in the \char`\"{}examples\char`\"{} and \char`\"{}test\char`\"{} directory of the Port\+Audio distribution. Another great source of info is the \hyperlink{portaudio_8h}{portaudio.\+h} Doxygen page, which documents the entire V19 A\+PI. Also see the page for \href{http://www.assembla.com/spaces/portaudio/wiki/Tips}{\tt tips on programming Port\+Audio} on the Port\+Audio wiki.\hypertarget{tutorial_start_tut_start4}{}\section{Programming Tutorial Contents}\label{tutorial_start_tut_start4}

\begin{DoxyItemize}
\item \hyperlink{writing_a_callback}{Writing a Callback Function}
\item \hyperlink{initializing_portaudio}{Initializing Port\+Audio}
\item \hyperlink{open_default_stream}{Opening a Stream Using Defaults}
\item \hyperlink{start_stop_abort}{Starting, Stopping and Aborting a Stream}
\item \hyperlink{terminating_portaudio}{Closing a Stream and Terminating Port\+Audio}
\item \hyperlink{utility_functions}{Utility Functions}
\item \hyperlink{querying_devices}{Enumerating and Querying Port\+Audio Devices}
\item \hyperlink{blocking_read_write}{Blocking Read/\+Write Functions}
\end{DoxyItemize}

If you are upgrading from V18, you may want to look at the \href{http://www.portaudio.com/docs/proposals/index.html}{\tt Proposed Enhancements to Port\+Audio}, which describes the differences between V18 and V19.

Once you have a basic understanding of how to use Port\+Audio, you might be interested in \hyperlink{exploring}{Exploring Port\+Audio}.

Next\+: \hyperlink{writing_a_callback}{Writing a Callback Function} 