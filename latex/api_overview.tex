This page provides a top-\/down overview of the entire Port\+Audio A\+PI. It describes how all of the Port\+Audio data types and functions fit together. It provides links to the documentation for each function and data type. You can find all of the detailed documentation for each A\+PI function and data type on the \hyperlink{portaudio_8h}{portaudio.\+h} page.\hypertarget{api_overview_introduction}{}\section{Introduction}\label{api_overview_introduction}
Port\+Audio provides a uniform application programming interface (A\+PI) across all supported platforms. You can think of the Port\+Audio library as a wrapper that converts calls to the Port\+Audio A\+PI into calls to platform-\/specific native audio A\+P\+Is. Operating systems often offer more than one native audio A\+PI and some A\+P\+Is (such as J\+A\+CK) may be available on multiple target operating systems. Port\+Audio supports all the major native audio A\+P\+Is on each supported platform. The diagram below illustrates the relationship between your application, Port\+Audio, and the supported native audio A\+P\+Is\+:



Port\+Audio provides a uniform interface to native audio A\+P\+Is. However, it doesn\textquotesingle{}t always provide totally uniform functionality. There are cases where Port\+Audio is limited by the capabilities of the underlying native audio A\+PI. For example, Port\+Audio doesn\textquotesingle{}t provide sample rate conversion if you request a sample rate that is not supported by the native audio A\+PI. Another example is that the A\+S\+IO S\+DK only allows one device to be open at a time, so Port\+Audio/\+A\+S\+IO doesn\textquotesingle{}t currently support opening multiple A\+S\+IO devices simultaneously.\hypertarget{api_overview_key_abstractions}{}\section{Key abstractions\+: Host A\+P\+Is, Devices and Streams}\label{api_overview_key_abstractions}
The Port\+Audio processing model includes three main abstractions\+: {\itshape Host A\+P\+Is}, audio {\itshape Devices} and audio {\itshape Streams}.

Host A\+P\+Is represent platform-\/specific native audio A\+P\+Is. Some examples of Host A\+P\+Is are Core Audio on Mac OS, W\+M\+ME and Direct\+Sound on Windows and O\+SS and A\+L\+SA on Linux. The diagram in the previous section shows many of the supported native A\+P\+Is. Sometimes it\textquotesingle{}s useful to know which Host A\+P\+Is you\textquotesingle{}re dealing with, but it is easy to use Port\+Audio without ever interacting directly with the Host A\+PI abstraction.

Devices represent individual hardware audio interfaces or audio ports on the host platform. Devices have names and certain capabilities such as supported sample rates and the number of supported input and output channels. Port\+Audio provides functions to enumerate available Devices and to query for Device capabilities.

Streams manage active audio input and output from and to Devices. Streams may be half duplex (input or output) or full duplex (simultaneous input and output). Streams operate at a specific sample rate with particular sample formats, buffer sizes and internal buffering latencies. You specify these parameters when you open the Stream. Audio data is communicated between a Stream and your application via a user provided asynchronous callback function or by invoking synchronous read and write functions.

Port\+Audio supports audio input and output in a variety of sample formats\+: 8, 16, 24 and 32 bit integer formats and 32 bit floating point, irrespective of the formats supported by the native audio A\+PI. Port\+Audio also supports multichannel buffers in both interleaved and non-\/interleaved (separate buffer per channel) formats and automatically performs conversion when necessary. If requested, Port\+Audio can clamp out-\/of range samples and/or dither to a native format.

The Port\+Audio A\+PI offers the following functionality\+:
\begin{DoxyItemize}
\item Initialize and terminate the library
\item Enumerate available Host A\+P\+Is
\item Enumerate available Devices either globally, or within each Host A\+PI
\item Discover default or recommended Devices and Device settings
\item Discover Device capabilities such as supported audio data formats and sample rates
\item Create and control audio Streams to acquire audio from and output audio to Devices
\item Provide Stream timing information to support synchronising audio with other parts of your application
\item Retrieve version and error information.
\end{DoxyItemize}

These functions are described in more detail below.\hypertarget{api_overview_top_level_functions}{}\section{Initialization, termination and utility functions}\label{api_overview_top_level_functions}
The Port\+Audio library must be initialized before it can be used and terminated to clean up afterwards. You initialize Port\+Audio by calling \hyperlink{portaudio_8h_abed859482d156622d9332dff9b2d89da}{Pa\+\_\+\+Initialize()} and clean up by calling \hyperlink{portaudio_8h_a0db317604e916e8bd6098e60e6237221}{Pa\+\_\+\+Terminate()}.

You can query Port\+Audio for version information using \hyperlink{portaudio_8h_a66da08bcf908e0849c62a6b47f50d7b4}{Pa\+\_\+\+Get\+Version()} to get a numeric version number and \hyperlink{portaudio_8h_a28f3fd9e6d9f933cc695abea71c4b445}{Pa\+\_\+\+Get\+Version\+Text()} to get a string.

The size in bytes of the various sample formats represented by the \hyperlink{portaudio_8h_a4582d93c2c2e60e12be3d74c5fe00b96}{Pa\+Sample\+Format} enumeration can be obtained using \hyperlink{portaudio_8h_a541ed0b734df2631bc4c229acf92abc1}{Pa\+\_\+\+Get\+Sample\+Size()}.

\hyperlink{portaudio_8h_a1b3c20044c9401c42add29475636e83d}{Pa\+\_\+\+Sleep()} sleeps for a specified number of milliseconds. This isn\textquotesingle{}t intended for use in production systems; it\textquotesingle{}s provided only as a simple portable way to implement tests and examples where the main thread sleeps while audio is acquired or played by an asynchronous callback function.\hypertarget{api_overview_host_apis}{}\section{Host A\+P\+Is}\label{api_overview_host_apis}
A Host A\+PI acts as a top-\/level grouping for all of the Devices offered by a single native platform audio A\+PI. Each Host A\+PI has a unique type identifier, a name, zero or more Devices, and nominated default input and output Devices.

Host A\+P\+Is are usually referenced by index\+: an integer of type \hyperlink{portaudio_8h_aeef6da084c57c70aa94be97411e19930}{Pa\+Host\+Api\+Index} that ranges between zero and \hyperlink{portaudio_8h_a19dbdb7c8702e3f4bfc0cdb99dac3dd9}{Pa\+\_\+\+Get\+Host\+Api\+Count()} -\/ 1. You can enumerate all available Host A\+P\+Is by counting across this range.

You can retrieve the index of the default Host A\+PI by calling \hyperlink{portaudio_8h_ae55c77f9b7e3f8eb301a6f1c0e2347ac}{Pa\+\_\+\+Get\+Default\+Host\+Api()}.

Information about a Host A\+PI, such as it\textquotesingle{}s name and default devices, is stored in a \hyperlink{struct_pa_host_api_info}{Pa\+Host\+Api\+Info} structure. You can retrieve a pointer to a particular Host A\+PI\textquotesingle{}s \hyperlink{struct_pa_host_api_info}{Pa\+Host\+Api\+Info} structure by calling \hyperlink{portaudio_8h_a7c650aede88ea553066bab9bbe97ea90}{Pa\+\_\+\+Get\+Host\+Api\+Info()} with the Host A\+PI\textquotesingle{}s index as a parameter.

Most Port\+Audio functions reference Host A\+P\+Is by \hyperlink{portaudio_8h_aeef6da084c57c70aa94be97411e19930}{Pa\+Host\+Api\+Index} indices. Each Host A\+PI also has a unique type identifier defined in the \hyperlink{portaudio_8h_ae247ec252e84112170079ece319fc42c}{Pa\+Host\+Api\+Type\+Id} enumeration. You can call \hyperlink{portaudio_8h_a081c3975126d20b4226facfb7ba0620f}{Pa\+\_\+\+Host\+Api\+Type\+Id\+To\+Host\+Api\+Index()} to retrieve the current \hyperlink{portaudio_8h_aeef6da084c57c70aa94be97411e19930}{Pa\+Host\+Api\+Index} for a particular \hyperlink{portaudio_8h_ae247ec252e84112170079ece319fc42c}{Pa\+Host\+Api\+Type\+Id}.\hypertarget{api_overview_devices}{}\section{Devices}\label{api_overview_devices}
A Device represents an audio endpoint provided by a particular native audio A\+PI. This usually corresponds to a specific input or output port on a hardware audio interface, or to the interface as a whole. Each Host A\+PI operates independently, so a single physical audio port may be addressable via different Devices exposed by different Host A\+P\+Is.

A Device has a name, is associated with a Host A\+PI, and has a maximum number of supported input and output channels. Port\+Audio provides recommended default latency values and a default sample rate for each Device. To obtain more detailed information about device capabilities you can call \hyperlink{portaudio_8h_abdb313743d6efef26cecdae787a2bd3d}{Pa\+\_\+\+Is\+Format\+Supported()} to query whether it is possible to open a Stream using particular Devices, parameters and sample rate.

Although each Device conceptually belongs to a specific Host A\+PI, most Port\+Audio functions and data structures refer to Devices using a global, Host A\+P\+I-\/independent index of type \hyperlink{portaudio_8h_ad79317e65bde63d76c4b8e711ac5a361}{Pa\+Device\+Index} -- an integer of that ranges between zero and \hyperlink{portaudio_8h_acfe4d3c5ec1a343f459981bfa2057f8d}{Pa\+\_\+\+Get\+Device\+Count()} -\/ 1. The reasons for this are partly historical but it also makes it easy for applications to ignore the Host A\+PI abstraction and just work with Devices and Streams.

If you want to enumerate Devices belonging to a particular Host A\+PI you can count between 0 and \hyperlink{struct_pa_host_api_info_a44e3adfaba0117a6780e2493468c96b1}{Pa\+Host\+Api\+Info\+::device\+Count} -\/ 1. You can convert this Host A\+P\+I-\/specific index value to a global \hyperlink{portaudio_8h_ad79317e65bde63d76c4b8e711ac5a361}{Pa\+Device\+Index} value by calling \hyperlink{portaudio_8h_a54f306b5e5258323c95a27c5722258cd}{Pa\+\_\+\+Host\+Api\+Device\+Index\+To\+Device\+Index()}.

Information about a Device is stored in a \hyperlink{struct_pa_device_info}{Pa\+Device\+Info} structure. You can retrieve a pointer to a Devices\textquotesingle{}s \hyperlink{struct_pa_device_info}{Pa\+Device\+Info} structure by calling \hyperlink{portaudio_8h_ac7d8e091ffc1d1d4a035704660e117eb}{Pa\+\_\+\+Get\+Device\+Info()} with the Device\textquotesingle{}s index as a parameter.

You can retrieve the indices of the global default input and output devices using \hyperlink{portaudio_8h_abf9f2f82da95553d5adb929af670f74b}{Pa\+\_\+\+Get\+Default\+Input\+Device()} and \hyperlink{portaudio_8h_adc955dfab007624000695c48d4f876dc}{Pa\+\_\+\+Get\+Default\+Output\+Device()}. Default Devices for each Host A\+PI are stored in the Host A\+PI\textquotesingle{}s \hyperlink{struct_pa_host_api_info}{Pa\+Host\+Api\+Info} structures.

For an example of enumerating devices and printing information about their capabilities see the pa\+\_\+devs.\+c program in the test directory of the Port\+Audio distribution.\hypertarget{api_overview_streams}{}\section{Streams}\label{api_overview_streams}
A Stream represents an active flow of audio data between your application and one or more audio Devices. A Stream operates at a specific sample rate with specific sample formats and buffer sizes.\hypertarget{api_overview_io_methods}{}\subsection{I/\+O Methods\+: callback and read/write}\label{api_overview_io_methods}
Port\+Audio offers two methods for communicating audio data between an open Stream and your Application\+: (1) an asynchronous callback interface, where Port\+Audio calls a user defined callback function when new audio data is available or required, and (2) synchronous read and write functions which can be used in a blocking or non-\/blocking manner. You choose between the two methods when you open a Stream. The two methods are discussed in more detail below.\hypertarget{api_overview_opening_and_closing_streams}{}\subsection{Opening and Closing Streams}\label{api_overview_opening_and_closing_streams}
You call \hyperlink{portaudio_8h_a443ad16338191af364e3be988014cbbe}{Pa\+\_\+\+Open\+Stream()} to open a Stream, specifying the Device(s) to use, the number of input and output channels, sample formats, suggested latency values and flags that control dithering, clipping and overflow handling. You specify many of these parameters in two \hyperlink{struct_pa_stream_parameters}{Pa\+Stream\+Parameters} structures, one for input and one for output. If you\textquotesingle{}re using the callback I/O method you also pass a callback buffer size, callback function pointer and user data pointer.

Devices may be full duplex (supporting simultaneous input and output) or half duplex (supporting input or output) -- usually this reflects the structure of the underlying native audio A\+PI. When opening a Stream you can specify one full duplex Device for both input and output, or two different Devices for input and output. Some Host A\+P\+Is only support full-\/duplex operation with a full-\/duplex device (e.\+g. A\+S\+IO) but most are able to aggregate two half duplex devices into a full duplex Stream. Port\+Audio requires that all devices specified in a call to \hyperlink{portaudio_8h_a443ad16338191af364e3be988014cbbe}{Pa\+\_\+\+Open\+Stream()} belong to the same Host A\+PI.

A successful call to \hyperlink{portaudio_8h_a443ad16338191af364e3be988014cbbe}{Pa\+\_\+\+Open\+Stream()} creates a pointer to a \hyperlink{portaudio_8h_a19874734f89958fccf86785490d53b4c}{Pa\+Stream} -- an opaque handle representing the open Stream. All Port\+Audio A\+PI functions that operate on open Streams take a pointer to a \hyperlink{portaudio_8h_a19874734f89958fccf86785490d53b4c}{Pa\+Stream} as their first parameter.

Port\+Audio also provides \hyperlink{portaudio_8h_a0a12735ac191200f696a43b87667b714}{Pa\+\_\+\+Open\+Default\+Stream()} -- a simpler alternative to \hyperlink{portaudio_8h_a443ad16338191af364e3be988014cbbe}{Pa\+\_\+\+Open\+Stream()} which you can use when you want to open the default audio Device(s) with default latency parameters.

You call \hyperlink{portaudio_8h_a92f56f88cbd14da0e8e03077e835d104}{Pa\+\_\+\+Close\+Stream()} to close a Stream when you\textquotesingle{}ve finished using it.\hypertarget{api_overview_starting_and_stopping_streams}{}\subsection{Starting and Stopping Streams}\label{api_overview_starting_and_stopping_streams}
Newly opened Streams are initially stopped. You call \hyperlink{portaudio_8h_a7432aadd26c40452da12fa99fc1a047b}{Pa\+\_\+\+Start\+Stream()} to start a Stream. You can stop a running Stream using \hyperlink{portaudio_8h_af18dd60220251286c337631a855e38a0}{Pa\+\_\+\+Stop\+Stream()} or \hyperlink{portaudio_8h_a138e57abde4e833c457b64895f638a25}{Pa\+\_\+\+Abort\+Stream()} (the Stop function plays out all internally queued audio data, while Abort tries to stop as quickly as possible). An open Stream can be started and stopped multiple times. You can call \hyperlink{portaudio_8h_a52d778c985ae9d566de7e13529cc771f}{Pa\+\_\+\+Is\+Stream\+Stopped()} to query whether a Stream is running or stopped.

By calling \hyperlink{portaudio_8h_aa11e7b06b2cde8621551f5d527965838}{Pa\+\_\+\+Set\+Stream\+Finished\+Callback()} it is possible to register a special \hyperlink{portaudio_8h_ab2530ee0cb756c67726f9074d3482ef2}{Pa\+Stream\+Finished\+Callback} that will be called when the Stream has completed playing any internally queued buffers. This can be used in conjunction with the \hyperlink{portaudio_8h_ae9bfb9c4e1895326f30f80d415c66c32aa01800db52ead393d8b0016f63d76650}{pa\+Complete} stream callback return value (see below) to avoid blocking on a call to \hyperlink{portaudio_8h_af18dd60220251286c337631a855e38a0}{Pa\+\_\+\+Stop\+Stream()} while queued audio data is still playing.\hypertarget{api_overview_callback_io_method}{}\subsection{The Callback I/\+O Method}\label{api_overview_callback_io_method}
So-\/called \textquotesingle{}callback Streams\textquotesingle{} operate by periodically invoking a callback function you supply to \hyperlink{portaudio_8h_a443ad16338191af364e3be988014cbbe}{Pa\+\_\+\+Open\+Stream()}. The callback function must implement the \hyperlink{portaudio_8h_a8a60fb2a5ec9cbade3f54a9c978e2710}{Pa\+Stream\+Callback} signature. It gets called by Port\+Audio every time Port\+Audio needs your application to consume or produce audio data. The callback is passed pointers to buffers containing the audio to process. The format (interleave, sample data type) and size of these buffers is determined by the parameters passed to \hyperlink{portaudio_8h_a443ad16338191af364e3be988014cbbe}{Pa\+\_\+\+Open\+Stream()} when the Stream was opened.

Stream callbacks usually return \hyperlink{portaudio_8h_ae9bfb9c4e1895326f30f80d415c66c32acba49cbf0e3bf605bb3deecef3b4fce3}{pa\+Continue} to indicate that Port\+Audio should keep the stream running. It is possible to deactivate a Stream from the stream callback by returning either \hyperlink{portaudio_8h_ae9bfb9c4e1895326f30f80d415c66c32aa01800db52ead393d8b0016f63d76650}{pa\+Complete} or \hyperlink{portaudio_8h_ae9bfb9c4e1895326f30f80d415c66c32acaf141eb5d71420ffb4474da1dbd8de2}{pa\+Abort}. In this case the Stream enters a deactivated state after the last buffer has finished playing (\hyperlink{portaudio_8h_ae9bfb9c4e1895326f30f80d415c66c32aa01800db52ead393d8b0016f63d76650}{pa\+Complete}) or as soon as possible (\hyperlink{portaudio_8h_ae9bfb9c4e1895326f30f80d415c66c32acaf141eb5d71420ffb4474da1dbd8de2}{pa\+Abort}). You can detect the deactivated state by calling \hyperlink{portaudio_8h_a1f8709c4971932643681a6f374c4bb5a}{Pa\+\_\+\+Is\+Stream\+Active()} or by using \hyperlink{portaudio_8h_aa11e7b06b2cde8621551f5d527965838}{Pa\+\_\+\+Set\+Stream\+Finished\+Callback()} to subscribe to a stream finished notification. Note that even if the stream callback returns \hyperlink{portaudio_8h_ae9bfb9c4e1895326f30f80d415c66c32aa01800db52ead393d8b0016f63d76650}{pa\+Complete} it\textquotesingle{}s still necessary to call \hyperlink{portaudio_8h_af18dd60220251286c337631a855e38a0}{Pa\+\_\+\+Stop\+Stream()} or \hyperlink{portaudio_8h_a138e57abde4e833c457b64895f638a25}{Pa\+\_\+\+Abort\+Stream()} to enter the stopped state.

Many of the tests in the /tests directory of the Port\+Audio distribution implement Port\+Audio stream callbacks. For example see\+: patest\+\_\+sine.\+c (audio output), patest\+\_\+record.\+c (audio input), patest\+\_\+wire.\+c (audio pass-\/through) and pa\+\_\+fuzz.\+c (simple audio effects processing).

{\bfseries I\+M\+P\+O\+R\+T\+A\+NT\+:} The stream callback function often needs to operate with very high or real-\/time priority. As a result there are strict requirements placed on the type of code that can be executed in a stream callback. In general this means avoiding any code that might block, including\+: acquiring locks, calling OS A\+PI functions including allocating memory. With the exception of \hyperlink{portaudio_8h_a83b8c624464dd7bb6a01b06ab596c115}{Pa\+\_\+\+Get\+Stream\+Cpu\+Load()} you may not call Port\+Audio A\+PI functions from within the stream callback.\hypertarget{api_overview_read_write_io_method}{}\subsection{The Read/\+Write I/\+O Method}\label{api_overview_read_write_io_method}
As an alternative to the callback I/O method, Port\+Audio provides a synchronous read/write interface for acquiring and playing audio. This can be useful for applications that don\textquotesingle{}t require the lowest possibly latency, or don\textquotesingle{}t warrant the increased complexity of synchronising with an asynchronous callback funciton. This I/O method is also useful when calling Port\+Audio from programming languages that don\textquotesingle{}t support asynchronous callbacks.

To open a Stream in read/write mode you pass a N\+U\+LL stream callback function pointer to \hyperlink{portaudio_8h_a443ad16338191af364e3be988014cbbe}{Pa\+\_\+\+Open\+Stream()}.

To write audio data to a Stream call \hyperlink{portaudio_8h_a075a6efb503a728213bdae24347ed27d}{Pa\+\_\+\+Write\+Stream()} and to read data call \hyperlink{portaudio_8h_a0b62d4b74b5d3d88368e9e4c0b8b2dc7}{Pa\+\_\+\+Read\+Stream()}. These functions will block if the internal buffers are full, making them safe to call in a tight loop. If you want to avoid blocking you can query the amount of available read or write space using \hyperlink{portaudio_8h_ad04c33f045fa58d7b705b56b1fd3e816}{Pa\+\_\+\+Get\+Stream\+Read\+Available()} or \hyperlink{portaudio_8h_a25595acf48733ec32045aa189c3caa61}{Pa\+\_\+\+Get\+Stream\+Write\+Available()} and use the returned values to limit the amount of data you read or write.

For examples of the read/write I/O method see the following examples in the /tests directory of the Port\+Audio distribution\+: patest\+\_\+read\+\_\+record.\+c (audio input), patest\+\_\+write\+\_\+sine.\+c (audio output), patest\+\_\+read\+\_\+write\+\_\+wire.\+c (audio pass-\/through).\hypertarget{api_overview_stream_info}{}\subsection{Retrieving Stream Information}\label{api_overview_stream_info}
You can retrieve information about an open Stream by calling \hyperlink{portaudio_8h_a3d9c4cbda4e9f381b76f287c3de8a758}{Pa\+\_\+\+Get\+Stream\+Info()}. This returns a \hyperlink{struct_pa_stream_info}{Pa\+Stream\+Info} structure containing the actual input and output latency and sample rate of the stream. It\textquotesingle{}s possible for these values to be different from the suggested values passed to \hyperlink{portaudio_8h_a443ad16338191af364e3be988014cbbe}{Pa\+\_\+\+Open\+Stream()}.

When using a callback stream you can call \hyperlink{portaudio_8h_a83b8c624464dd7bb6a01b06ab596c115}{Pa\+\_\+\+Get\+Stream\+Cpu\+Load()} to retrieve a rough estimate of the amount of C\+PU time your callback function is using.\hypertarget{api_overview_stream_timing}{}\subsection{Stream Timing Information}\label{api_overview_stream_timing}
When using the callback I/O method your stream callback function receives timing information via a pointer to a \hyperlink{struct_pa_stream_callback_time_info}{Pa\+Stream\+Callback\+Time\+Info} structure. This structure contains the current time along with the estimated hardware capture and playback time of the first sample of the input and output buffers. All times are measured in seconds relative to a Stream-\/specific clock. The current Stream clock time can be retrieved using \hyperlink{portaudio_8h_a2b3fb60e6949f37f7f134105ff425749}{Pa\+\_\+\+Get\+Stream\+Time()}.

You can use the stream callback \hyperlink{struct_pa_stream_callback_time_info}{Pa\+Stream\+Callback\+Time\+Info} times in conjunction with timestamps returned by \hyperlink{portaudio_8h_a2b3fb60e6949f37f7f134105ff425749}{Pa\+\_\+\+Get\+Stream\+Time()} to implement time synchronization schemes such as time aligning your G\+UI display with rendered audio, or maintaining synchronization between M\+I\+DI and audio playback.\hypertarget{api_overview_error_handling}{}\section{Error Handling}\label{api_overview_error_handling}
Most Port\+Audio functions return error codes using values from the \hyperlink{portaudio_8h_a4949e4a8ef9f9dbe8cbee414ce69841d}{Pa\+Error} enumeration. All error codes are negative values. Some functions return values greater than or equal to zero for normal results and a negative error code in case of error.

You can convert \hyperlink{portaudio_8h_a4949e4a8ef9f9dbe8cbee414ce69841d}{Pa\+Error} error codes to human readable text by calling \hyperlink{portaudio_8h_ae606855a611cf29c7d2d7421df5e3b5d}{Pa\+\_\+\+Get\+Error\+Text()}.

Port\+Audio usually tries to translate error conditions into portable \hyperlink{portaudio_8h_a4949e4a8ef9f9dbe8cbee414ce69841d}{Pa\+Error} error codes. However if an unexpected error is encountered the \hyperlink{portaudio_8h_a2e45bf8b5145f131a91c128af2bdaec7a47726071f5dccc656d5e3ff20bbfc5a0}{pa\+Unanticipated\+Host\+Error} code may be returned. In this case a further mechanism is provided to query for Host A\+P\+I-\/specific error information. If Port\+Audio returns \hyperlink{portaudio_8h_a2e45bf8b5145f131a91c128af2bdaec7a47726071f5dccc656d5e3ff20bbfc5a0}{pa\+Unanticipated\+Host\+Error} you can call \hyperlink{portaudio_8h_aad573f208b60577f21d2777a7c5054e0}{Pa\+\_\+\+Get\+Last\+Host\+Error\+Info()} to retrieve a pointer to a \hyperlink{struct_pa_host_error_info}{Pa\+Host\+Error\+Info} structure that provides more information, including the Host A\+PI that encountered the error, a native A\+PI error code and error text.\hypertarget{api_overview_host_api_extensions}{}\section{Host A\+P\+I and Platform-\/specific Extensions}\label{api_overview_host_api_extensions}
The public Port\+Audio A\+PI only exposes functionality that can be provided across all target platforms. In some cases individual native audio A\+P\+Is offer unique functionality. Some Port\+Audio Host A\+P\+Is expose this functionality via Host A\+P\+I-\/specific extensions. Examples include access to low-\/level buffering and priority parameters, opening a Stream with only a subset of a Device\textquotesingle{}s channels, or accessing channel metadata such as channel names.

Host A\+P\+I-\/specific extensions are provided in the form of additional functions and data structures defined in Host A\+P\+I-\/specific header files found in the /include directory.

The \hyperlink{struct_pa_stream_parameters}{Pa\+Stream\+Parameters} structure passed to \hyperlink{portaudio_8h_abdb313743d6efef26cecdae787a2bd3d}{Pa\+\_\+\+Is\+Format\+Supported()} and \hyperlink{portaudio_8h_a443ad16338191af364e3be988014cbbe}{Pa\+\_\+\+Open\+Stream()} has a field named \hyperlink{struct_pa_stream_parameters_aff01b9fa0710ad1654471e97665c06a9}{Pa\+Stream\+Parameters\+::host\+Api\+Specific\+Stream\+Info} that is sometimes used to pass low level information when opening a Stream.

See the documentation for the individual Host A\+P\+I-\/specific header files for details of the extended functionality they expose\+:


\begin{DoxyItemize}
\item \hyperlink{pa__asio_8h}{pa\+\_\+asio.\+h}
\item \hyperlink{pa__jack_8h}{pa\+\_\+jack.\+h}
\item \hyperlink{pa__linux__alsa_8h}{pa\+\_\+linux\+\_\+alsa.\+h}
\item \hyperlink{pa__mac__core_8h}{pa\+\_\+mac\+\_\+core.\+h}
\item \hyperlink{pa__win__ds_8h}{pa\+\_\+win\+\_\+ds.\+h}
\item \hyperlink{pa__win__wasapi_8h}{pa\+\_\+win\+\_\+wasapi.\+h}
\item \hyperlink{pa__win__wmme_8h}{pa\+\_\+win\+\_\+wmme.\+h}
\item \hyperlink{pa__win__waveformat_8h}{pa\+\_\+win\+\_\+waveformat.\+h} 
\end{DoxyItemize}