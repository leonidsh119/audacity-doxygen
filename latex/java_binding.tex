{\itshape Note\+: this page has not been reviewed, and may contain errors.}\hypertarget{java_binding_java_draft}{}\section{D\+R\+A\+F\+T -\/ I\+N P\+R\+O\+G\+R\+E\+SS}\label{java_binding_java_draft}
9/4/12 J\+Port\+Audio is very new and should be considered an \char`\"{}alpha\char`\"{} release. The building of J\+Port\+Audio will eventually be integrated into the Makefile as an optional build.

Currently J\+Port\+Audio is only supported for Windows and Macintosh. Please contact us if you want to help with porting Linux.

For reference documentation of the J\+Port\+Audio A\+PI see\+: \hyperlink{classcom_1_1portaudio_1_1_port_audio}{com.\+portaudio.\+Port\+Audio}

For an example see\+: \hyperlink{_play_sine_8java}{Play\+Sine.\+java}\hypertarget{java_binding_java_comp_windows}{}\section{Building J\+Port\+Audio on Windows}\label{java_binding_java_comp_windows}
Build the Java code using the Eclipse project in \char`\"{}jportaudio\char`\"{}. \hyperlink{class_export}{Export} as \char`\"{}jportaudio.\+jar\char`\"{}.

If you modify the J\+NI A\+PI then you will need to regenerate the J\+NI .h files using\+:


\begin{DoxyCode}
\hyperlink{conv2dtest_8m_a51ec853512a386583838922288834695}{cd} bindings/java/\hyperlink{namespacesetup_ac1f45f8d37050b278bf63c812b1130dd}{scripts}
make\_header.bat
\end{DoxyCode}


Build the J\+NI D\+LL using the Visual Studio 2010 solution in \char`\"{}java/c/build/vs2010/\+Port\+Audio\+J\+N\+I\char`\"{}.\hypertarget{java_binding_java_use_windows}{}\section{Using J\+Port\+Audio on Windows}\label{java_binding_java_use_windows}
Put the \char`\"{}jportaudio.\+jar\char`\"{} in the classpath for your application. Place the following libraries where they can be found, typically in the same folder as your application.


\begin{DoxyItemize}
\item portaudio\+\_\+x86.\+dll
\item portaudio\+\_\+x64.\+dll
\item jportaudio\+\_\+x86.\+dll
\item jportaudio\+\_\+x64.\+dll
\end{DoxyItemize}\hypertarget{java_binding_java_comp_max}{}\section{Building J\+Port\+Audio on Mac}\label{java_binding_java_comp_max}
These are notes from building J\+Port\+Audio on a Mac with 10.\+6.\+8 and X\+Code 4.

I created a target of type \textquotesingle{}C\textquotesingle{} library.

I added the regular Port\+Audio frameworks plus the Java\+VM framework.

I modified \hyperlink{com__portaudio___port_audio_8h}{com\+\_\+portaudio\+\_\+\+Port\+Audio.\+h} and \hyperlink{com__portaudio___blocking_stream_8h}{com\+\_\+portaudio\+\_\+\+Blocking\+Stream.\+h} so that jni.\+h could found.


\begin{DoxyCode}
\textcolor{preprocessor}{#if defined(\_\_APPLE\_\_)}
\textcolor{preprocessor}{#include <JavaVM/jni.h>}
\textcolor{preprocessor}{#else}
\textcolor{preprocessor}{#include <jni.h>}
\textcolor{preprocessor}{#endif}
\end{DoxyCode}


This is bad because those header files are autogenerated and will be overwritten. We need a better solution for this.

I had trouble finding the \char`\"{}libjportaudio.\+jnilib\char`\"{}. So I added a Build Phase that copied the library to \char`\"{}/\+Users/phil/\+Library/\+Java/\+Extensions\char`\"{}.

On the Mac we can create a universal library for both 32 and 64-\/bit J\+V\+Ms. So in the J\+AR file I will open \char`\"{}jportaudio\char`\"{} on Apple. ON W\+Indows I will continue to open \char`\"{}jportaudio\+\_\+x64\char`\"{} and \char`\"{}jportaudio\+\_\+x86\char`\"{}. 