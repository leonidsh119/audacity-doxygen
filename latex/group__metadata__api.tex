\hypertarget{group__metadata__api}{}\section{Public Metadata A\+PI}
\label{group__metadata__api}\index{Public Metadata A\+PI@{Public Metadata A\+PI}}
The metadata A\+PI allows libavformat to export metadata tags to a client application when demuxing. Conversely it allows a client application to set metadata when muxing.

Metadata is exported or set as pairs of key/value strings in the \textquotesingle{}metadata\textquotesingle{} fields of the \hyperlink{struct_a_v_format_context}{A\+V\+Format\+Context}, \hyperlink{struct_a_v_stream}{A\+V\+Stream}, \hyperlink{struct_a_v_chapter}{A\+V\+Chapter} and \hyperlink{struct_a_v_program}{A\+V\+Program} structs using the \hyperlink{group__lavu__dict}{A\+V\+Dictionary} A\+PI. Like all strings in F\+Fmpeg, metadata is assumed to be U\+T\+F-\/8 encoded Unicode. Note that metadata exported by demuxers isn\textquotesingle{}t checked to be valid U\+T\+F-\/8 in most cases.

Important concepts to keep in mind\+:
\begin{DoxyItemize}
\item Keys are unique; there can never be 2 tags with the same key. This is also meant semantically, i.\+e., a demuxer should not knowingly produce several keys that are literally different but semantically identical. E.\+g., key=Author5, key=Author6. In this example, all authors must be placed in the same tag.
\item Metadata is flat, not hierarchical; there are no subtags. If you want to store, e.\+g., the email address of the child of producer Alice and actor Bob, that could have key=alice\+\_\+and\+\_\+bobs\+\_\+childs\+\_\+email\+\_\+address.
\item Several modifiers can be applied to the tag name. This is done by appending a dash character (\textquotesingle{}-\/\textquotesingle{}) and the modifier name in the order they appear in the list below -- e.\+g. foo-\/eng-\/sort, not foo-\/sort-\/eng.
\begin{DoxyItemize}
\item language -- a tag whose value is localized for a particular language is appended with the I\+SO 639-\/2/B 3-\/letter language code. For example\+: Author-\/ger=Michael, Author-\/eng=Mike The original/default language is in the unqualified \char`\"{}\+Author\char`\"{} tag. A demuxer should set a default if it sets any translated tag.
\item sorting -- a modified version of a tag that should be used for sorting will have \textquotesingle{}-\/sort\textquotesingle{} appended. E.\+g. artist=\char`\"{}\+The Beatles\char`\"{}, artist-\/sort=\char`\"{}\+Beatles, The\char`\"{}.
\end{DoxyItemize}
\item Demuxers attempt to export metadata in a generic format, however tags with no generic equivalents are left as they are stored in the container. Follows a list of generic tag names\+:
\end{DoxyItemize}

\begin{DoxyVerb}album        -- name of the set this work belongs to
album_artist -- main creator of the set/album, if different from artist.
                e.g. "Various Artists" for compilation albums.
artist       -- main creator of the work
comment      -- any additional description of the file.
composer     -- who composed the work, if different from artist.
copyright    -- name of copyright holder.
creation_time-- date when the file was created, preferably in ISO 8601.
date         -- date when the work was created, preferably in ISO 8601.
disc         -- number of a subset, e.g. disc in a multi-disc collection.
encoder      -- name/settings of the software/hardware that produced the file.
encoded_by   -- person/group who created the file.
filename     -- original name of the file.
genre        -- <self-evident>.
language     -- main language in which the work is performed, preferably
                in ISO 639-2 format. Multiple languages can be specified by
                separating them with commas.
performer    -- artist who performed the work, if different from artist.
                E.g for "Also sprach Zarathustra", artist would be "Richard
                Strauss" and performer "London Philharmonic Orchestra".
publisher    -- name of the label/publisher.
service_name     -- name of the service in broadcasting (channel name).
service_provider -- name of the service provider in broadcasting.
title        -- name of the work.
track        -- number of this work in the set, can be in form current/total.
variant_bitrate -- the total bitrate of the bitrate variant that the current stream is part of
\end{DoxyVerb}


Look in the examples section for an application example how to use the Metadata A\+PI. 